~

\newpage

~

\vfill

\begin{quote}
    \par\raggedleft
    \emph{Aux trois choses importantes dans la vie:} \\
    {\LARGE 友情と恋とモビルスーツ}
\end{quote}

\vfill\vfill

%\newpage

\chapter*{Remerciements}

\begin{quote}
    If I have seen further, it is by standing on the shoulders of giants.
    \par\raggedleft--- \emph{Isaac Newton}, 1675
\end{quote}

Cette formule, reprise depuis le XII\textsuperscript{e} siècle par maints scientifiques, n'a jamais cessé d'évoluer dans
mon esprit. À mon entrée au collège, je pensais qu'elle parlait de Thalès et Pythagore, et me demandais où nous en
serions aujourd'hui sans ces grands personnages du VI\textsuperscript{e} siècle avant J.-C.

Évidemment, cette liste s'est rapidement agrandie au fil de mes études: Galilée, Descartes, Newton, Tesla, Einstein,
Turing\ldots{} Mais en débutant cette thèse, je me suis rendu compte qu'au sein de l'équipe Gepetto du LAAS-CNRS, ces
géants qui nous permettent de voir plus loin, ils sont présents, et dépensent même du temps et de l'énergie à nous
hisser au-dessus de leurs épaules.

Je tiens donc à remercier grandement Jean-Paul, Michel, Nicolas, Florent, Olivier et Philippe pour ces trois années
d'aide, de bonne humeur, de sciences, de conseils avisés, de projets fous, voire même d'aventures que vous avez
dirigées.

La bonne ambiance au sein de cette équipe m'a permis de travailler dans d'excellentes conditions (et donc de bien
travailler), et m'a également motivé à y rester un peu plus pour la suite. Je souhaite donc remercier pour cela tous
les doctorants, post-doctorants et stagiaires que j'y ai rencontré, et notamment Mehdi, Naoko, Maximilien, Andrea,
Steve, Alexis, Bernard, Céline, Diane, Dinesh, François, Gabriele, Justin, Kévin, Mathieu, Mylène, Nassime, Nirmal,
Pierre et Thomas.

Aussi, il me faut également remercier du fond du cœur ma famille et mes amis, qui m'ont aidé, motivé, et soutenu,
souvent sans même s'en rendre compte. Clément, Daniel, Luc, et vos familles, merci pour tout pour ces 20 dernières
années.

Merci aussi au club informatique de l'INP-ENSEEIHT, net7, et en particulier à lionel, pierref, storm, YGA, xouillet,
zempashi, francor, benoit, seb, meloiso, bok, fabien, antoine, huhuh, vinduv, vins, tysebap, djanos, ethelward, ken,
nug, linkid, nouph, maxima, viod, palkeo, carlm, mak.aʁtɔ̃, CHA, gofish, kordump, patate, toffan, jayjader, sligoo,
pibou, kaname, zadig, et zil0, pour toutes les discussions et débats techniques ainsi que la vie au club durant ces
sept années.

Et enfin, évidemment, un immense merci à Delphine, qui m'a énormément aidé tout au long de ces années de thèse, à un
point que je n'imaginais même pas possible.
