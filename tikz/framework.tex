\documentclass{minimal}
\usepackage[utf8]{inputenc}
\usepackage{amsmath}
\usepackage{tikz}
\usetikzlibrary{calc,patterns,angles,quotes,shapes,shadows}
\usetikzlibrary{shapes.geometric,arrows.meta,decorations.markings,}
\newcommand{\muscod}{\mbox{MUSCOD-II}}

\begin{document}

\pgfmathsetmacro{\i}{30}
\pgfmathsetmacro{\no}{90 + \i}
\pgfmathsetmacro{\ne}{90 - \i}
\pgfmathsetmacro{\so}{270 - \i}
\pgfmathsetmacro{\se}{270 + \i}

\begin{tikzpicture}[auto,align=center]
    \node [draw,rectangle,fill=blue!10,minimum height=3.7cm,minimum width=3.6cm] (fw) {};
    \node [draw,rectangle,fill=blue!20,above of=fw,node distance=1cm,text width=2.5cm] (sol) {Solveur (\muscod)};
    \node [draw,rectangle,fill=blue!20,below of=fw,node distance=1cm,text width=2.5cm] (sim) {Simulateur (Pinocchio)};

    \node [node distance=4.5cm,align=right,right,left of=sim,text width=3.25cm] (model) {modèle \mbox{environement}};
    \node [node distance=4.5cm,align=right,right,left of=sol,text width=3.25cm] (cot) {\mbox{fonction de coût} contraintes
    \mbox{type d’actionnement}};
    %\node [node distance=2.5cm,left of=sol] (param) {$\bp$};
    \node [node distance=4.2cm,right of=sol,text width=2cm,align=left] (out) {coût \\trajectoire paramètres};

    \draw [->] (sol.\so) -- node [align=left,left] {contrôle} (sim.\no);
    \draw [->] (sim.\ne) -- node [align=right,right] {état} (sol.\se);)
    \draw [->] (model) -- (sim);
    %\draw [->] (param) -- (sol);
    \draw [->] (cot) -- (sol);
    \draw [->] (sol) -- (out);
\end{tikzpicture}
\end{document}
