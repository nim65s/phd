~

\newpage

~

\vfill

\begin{quote}
    \par\raggedleft
    La simplicité est l'ultime sophistication. \\
    --- \emph{Leonardo da Vinci}

\end{quote}

\vfill

\begin{quote}
    \par\raggedleft
    Aux trois choses importantes dans la vie: \\
    {\LARGE 友情と恋とモビルスーツ} \\
    L'amitié, l'amour, et les robots humanoïdes. \\
    --- \emph{Gundam AGE 17}
\end{quote}

\vfill

\newpage

~

\newpage

\section*{Remerciements}

\begin{quote}
    If I have seen further, it is by standing on the shoulders of giants.
    \par\raggedleft--- \emph{Isaac Newton}, 1675
\end{quote}

Cette formule, reprise depuis le XII\textsuperscript{e} siècle par maints scientifiques, n'a jamais cessé d'évoluer
dans mon esprit. À mon entrée au collège, je pensais qu'elle parlait de Thalès et Pythagore, et me demandais où nous
en serions aujourd'hui sans ces grands personnages du VI\textsuperscript{e} siècle avant J.-C.

Évidemment, cette liste s'est rapidement agrandie au fil de mes études: Galilée, Descartes, Newton, Tesla, Einstein,
Turing\ldots{} Mais en débutant cette thèse, je me suis rendu compte qu'au sein de l'équipe Gepetto du LAAS-CNRS, ces
géants qui nous permettent de voir plus loin peuvent également être présents à nos côtés, et même passer du temps et de
l'énergie à nous hisser eux-même au-dessus de leurs épaules.

Je tiens donc à remercier grandement Jean-Paul, Florent, Michel, Nicolas, Olivier et Philippe pour ces trois années
de discussions, de bonne humeur, de sciences, de conseils avisés, de solutions techniques, de projets fous, voire même
d'aventures que vous avez dirigées.

Je dois aussi remercier Mmes. Chevallereau et D'Andrea-Novel pour avoir accepté de rapporter cette thèse, et la société
BA Systèmes pour avoir travaillé avec nous sur les projets présentés dans les \cref{sec:lemon,sec:transhumus}, et
particulièrement à M. Caverot qui a accepté d'être membre du jury.

La bonne ambiance au sein de l'équipe m'a permis de travailler dans d'excellentes conditions, et m'a par ailleurs
motivé à essayer d'y rester pour la suite. Je souhaite donc remercier pour cela tous les doctorants, post-doctorants
et stagiaires que j'y ai rencontré, et notamment Alexis, Alexis, Andrea, Bernard, Céline, Diane, Dinesh, Florenç,
François, Gabriele, Justin, Kévin, Mathieu, Maximilien, Mehdi, Mylène, Naoko, Nassime, Nirmal, Pierre, Rohan, Steve et
Thomas.

Il me faut également remercier du fond du cœur ma famille et mes amis, qui m'ont aidé, motivé, et soutenu, parfois sans
même s'en rendre compte. Clément, Daniel, Luc, et vos familles, merci pour tout pour ces 20 dernières années.

Merci de même au club informatique de l'INP-ENSEEIHT, net7, et en particulier à lionel, pierref, storm, YGA, xouillet,
zempashi, francor, benoit, seb, meloiso, bok, fabien, antoine, huhuh, vinduv, vins, tysebap, djanos, ethelward, ken,
nug, linkid, nouph, maxima, viod, palkeo, carlm, mak.aʁtɔ̃, CHA, gofish, kordump, patate, toffan, jayjader, sligoo,
pibou, kaname, zadig, et zil0, pour toutes les discussions, l'entraide, les débats techniques, tout ce que j'ai pu
apprendre, ainsi que la vie au club durant ces 7 années.

Et enfin, évidemment, un immense merci à Delphine, qui m'a énormément aidé tout au long de cette thèse, à un point que
je n'imaginais pas possible, et dont je croyais être largement capable me passer. En fait, on l'a bien vu, non.
